\problem{}
\subproblem{}
 به سادگی مشخص است که این عبارت غلط است زیرا شخصی که از این رستوران سفارش داده است یا از قبل سفارش داده بوده
 که به وضوح راضی بوده و مشتری رستوران است و نظر او مثبت و در غیر این صورت این شخص از طرف شخصی دیگر
 که سلیقه های مشابهی دارند معرفی شده یا خودش در نگاه اول از رستوران راضی بوده که این رستوران را برای سفارش انتخاب کرده
 و در کل تعداد حالات کمی وجود دارد که شخص نظر منفی داشته باشد و این نمونه گیری به وضوح اریب است.
\newline
\subproblem{} 
این نمونه گیری نیز به دلایل مختلفی که چند نمونه را ذکر خواهم کرد اریب است. یکی اینکه ریاضی دو درسی پایه و اجباری است.
یکی دیگر اینکه درس فلسفه ریاضی فقط مربوط به دانشکده ریاضی و همجین درس اختیاری است در صورتی که 
ریاضی دو مربوط به کل دانشگاه و اجباری است.
یکی دیگر از دلایل این است که برای خیلی از دانشجویان نمره خوب را به سادگی گرفتن ملاک است ولی دکتر شهشهانی استادی
سخت گیر است.