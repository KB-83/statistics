\problem{}
\[ e = P(E) \]
\[ f = P(F) \]
\[ s = P((F \cup E)^{c}) = 1 - e - f \]

با تعریف احتمالات بالا و فرض مستقل بودن آزمایش‌ها، احتمال رخ دادن E قبل از F برابر است با مجموع تمام حالت‌هایی که مجموعه‌ای ازمایش رخ داده و نه F و نه E رخ داده است و در آخرین آزمایش E رخ داده است. از آنجایی که این پیشامد‌ها اشتراکی ندارند، احتمال اجتماع آنها جمع احتمالاتشان است. بنابراین:

\[ P(E \text{ قبل } F) = \sum_{i=0}^{n} (s^i \times e) = e \times \sum_{i=0}^{n} (s^i) \]

که یک سری هندسی است و جمع آن به صورت زیر حساب می‌شود:

\[ \lim_{n \to \infty} \left( \frac{(1 - (s^n))}{1 - s} \right) = \frac{1}{1 - 1 + e + f} \]

در نتیجه داریم:

\[ P(E \text{ قبل } F) = \frac{e}{f+e} = \frac{P(E)}{P(E)+P(F)} \]
