\problem{}


\subproblem{}
در این حالت هر استراتژی ما در نهایت به یک دنباله 
از شماره و خال کارت ها تقسیم میشود که وقتی هیچ اطلاعاتی ا
ز کارت ها نداریم، دنباله ما صرفا یک جایگشت رندوم از کل کارت ها است. پس برای محاسبه $E[N]$ می توانیم متغیر تصادفی $X_i$ را تعریف کنیم که بر
نولی درست بودن کارت $i$ ام را نشان می‌دهد.  
\\
\[ E[N] = \sum_{i=1}^{52}E[X_i] = 52 \times \frac{1}{52} = 1 \]

\subproblem{}
در این قسمت، در هر استراتژی، نهایت اطلاعات ما در مرحله $i$ ام این است که کارت های ۱ تا $i-1$ چه بوده اند که می‌توانیم از گفتن آنها برای کار
ت $i$ ام صرف نظر کنیم. پس دوباره مثل قسمت قبل می‌توانیم $X_i$ ها را تعریف کنیم و 
این بار داریم:
\[ E[N] = \sum_{i=1}^{n}E[X_i] = \sum_{i=1}^{n} \frac{1}{n - i + 1} \approx \ln{n} \]
