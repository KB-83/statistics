\problem{}
\[
  E[X] = \int_{b}^{\infty}{xf(x)dx} = 
  \frac{e^{\frac{b}{a}}}{a}\int_{b}^{\infty}{xe^{-x}} =
  \frac{(1+b)}{a}e^{\frac{b}{a}-b}
\]

از طرفی چون تابع چگالی است داریم:\\
\[
  1 = \int_{b}^{\infty}{f(x)dx} = 
  \frac{e^{\frac{b}{a}}}{a}\int_{b}^{\infty}{e^{-x}} =
  \frac{1}{a}e^{\frac{b}{a}-b}
\]
قرار میدهیم :\\
\[
    \bar{X} = \sum_{i = 0}^{n}{\frac{X_i}{n}}    
\]

 از طرفی دیگر داریم با روش گشتاور:\\
 \[
    \frac{(1+b)}{a}e^{\frac{b}{a}-b}  = \bar{X}
 \]

 می دانیم $\frac{1}{a}e^{\frac{b}{a}-b} = 1$
 پس در رابطه های بالا بجای این عبارت $1$ را قرار میدهیم داریم:\\
 \[
    1+b = \bar{X} \quad => \quad b = \bar{X} - 1
 \]
حالا باید $a$ را برحسب $b$ حل کنیم به صورتی که 
$\frac{1}{a}e^{\frac{b}{a}-b} = 1$
را حل کنیم که $a = 1$
در این معادله جواب میدهد و از آنجایی که تابع چگالی بر حسب a است و یکتاست
پس همین جواب کافیست و روش گشتاوری به ما برآورد گر های زیر را میدهد:\\
\[
    a = 1 \quad b = \bar{X} - 1
\]
توجه کنید که این معادله بر حسب $b = \bar{X} - 1$
 ممکن است جواب های دیگری نیز داشته باشد برای $a$
 اما یکی از جواب ها حتما همیشه $1$
 است که از داده مستقل است همچنین جواب دیگر را میتوان بر حسب داده به صورت ضمنی محاسبه کرد.
 که رابطه ضمنی ما هست:\\
 \[
    b = \bar{X} - 1 =\frac{a}{1-a}\log^{a}
 \]









%  \[
%     \frac{(b^2)}{a}e^{\frac{b}{a}-b} 
%     +2\frac{(b+1)}{a}e^{\frac{b}{a}-b} 
%     = \sigma^2
%  \]
%  \[
%     \frac{(b^2+2b+2)}{\sigma^2}e^{\frac{b}{a}-b} = a
%  \]
% %  \[
% %     \frac{(b^2)}{a}e^{\frac{b}{a}-b} = \sigma^2 - 2\bar{X}
% %  \]

%  \[
%     \frac{(b^2)}{a}e^{\frac{b}{a}-b} = \sigma^2 - 2\bar{X}
%  \]

%  \[
%     \frac{({\log^{a}\frac{a}{1-a}})^2}{a}e^{\frac{\log^{a}\frac{a}{1-a}}{a}-\log^{a}\frac{a}{1-a}} = \sigma^2 - 2\bar{X}
%  \]



% \[
%     \frac{E[X]}{E[X^2]} =  \frac{\bar{X}}{\sigma^2} = \frac{1+b}{b^2+2b+2}
% \]
% \[
%   E[X^2] = \int_{b}^{\infty}{x^2f(x)dx} = 
%   \frac{e^{\frac{b}{a}}}{a}\int_{b}^{\infty}{x^2e^{-x}} =
%   \frac{(b^2+2b+2)}{a}e^{\frac{b}{a}-b}
% \]

% و
% \[
%     \sigma^2 = \sum_{i = 0}^{n}{\frac{{X_i}^2}{n}}    
% \]
% \[
    % \frac{(b^2+2b+2)}{a}e^{\frac{b}{a}-b} = \sigma^2
%  \]