\problem{}
اگر فرض کنیم $p$
را با $\hat{p}$
بر آورد کنیم احتمال این رخداد برابر است با:\\
\[
    f(\hat{p}) = \hat{p}^k (1-\hat{p})^{n-k}
\]
حالا طبق این روش برای بدست آوردن ماکسیمم تابع بالا
 باید از تابع بالا نسبت به $\hat{p}$
مشتق بگیریم داریم:\\
\[
    f^{\prime}(\hat{p}) =  k\hat{p}^{k-1}(1-\hat{p})^{n-k} - (n-k)\hat{p}^k(1-\hat{p})^{n-k-1}
\]
اگر این عبارت را برابر صفر قرار دهیم داریم:\\
\[
    k(1-\hat{p}) =\hat{p}(n-k)
\]
که این روش پیشنهاد میدهد قرار دهیم$\hat{p} = \frac{k}{n} \quad$.\\\\
(به این نکته توجه کنید که این تابع در نقاط دو سر بازه
ماکسیمم نمیشود $f(0) = f(1) = 0$)