\problem{}
% پس از واکسن زدن اولیه
% تعداد جانداران به صورت زیر میشود:\\
% تعداد واکسن زده ها $= n_1$\\
% تعداد کل $= N$.\\
% حالا در مرحله دوم اگر برآورد گر ما $\hat{N}$
% باشد آنگاه 
% احتمال اینکه تعداد واکسن زده ها از $n_2$ نمونه برابر با $m$
% باشد به صورت زیر است:\\
اگر قرار دهیم $X =\text{تعداد جانداران واکسن زده دیده شده در مرحله دوم}$
\[
    P[X = m|\hat{N}] =f(\hat{N}) =  \frac{\binom{n_1}{m}\binom{\hat{N}-n_1}{n_2-m}}{\binom{\hat{N}}{n_2}}  
\]
برای ماکسیمم کردن این عبارت در نظر میگیریم که تابع
$f(\hat{N})$ تا کجا صعودیست 
که یعنی تا کجا \\
$f(\hat{N+1}) > f(\hat{N})$
است.\\
پس میخواهیم:\\
\[
    \frac{f(\hat{N+1})}{f(\hat{N})} \geq 1
\]
\[
    \frac{\binom{\hat{N}}{n_2}\binom{n_1}{m}\binom{\hat{N}+1-n_1}{n_2-m}}
    {\binom{n_1}{m}\binom{\hat{N}-n_1}{n_2-m}\binom{\hat{N}+1}{n_2}} = 
    \frac{\binom{\hat{N}}{n_2}\binom{\hat{N}+1-n_1}{n_2-m}}
    {\binom{\hat{N}-n_1}{n_2-m}\binom{\hat{N}+1}{n_2}}  \geq 1
\]\\
\[
    \frac{\hat{N}!(\hat{N}+1-n_2)!(\hat{N}+1-n_1)!(\hat{N}-n_1-n_2+m)!}
    {(\hat{N}-n_2)!(\hat{N}+1)!(\hat{N}-n_1)!(\hat{N}+1-n_1-n_2+m)!} \geq1
\]\\
\[
    \frac{(\hat{N}+1-n_2)(\hat{N}+1-n_2)}
    {(\hat{N}+1)(\hat{N}+1-n_1-n_2+m)} \geq1
\]
قرار میدهیم $\hat{N}+1 = x$داریم:\\
\[
    \frac{(x-n_2)(x-n_1)}
    {x(x-n_1-n_2+m)} \geq 1
\]
\\
\[
    x^2-n_2x-n_1x+n_1n_2 \geq x^2-n_2x-n_1x+mx 
\]\\
\[
    \frac{n_1n_2}{m}\geq \hat{N}+1
\]\\
و اولین جایی که این رابطه نقض میشود جواب ماست که هست:\\
\[
    \hat{N} = \frac{n_1n_2}{m}  
\]
که همان برآوردگر پیشنهادی M.L.E است.