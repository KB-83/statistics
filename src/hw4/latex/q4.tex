\problem{}
\subproblem{}
\[
    f(x|\hat{\theta}) =
    \begin{cases}
        \frac{1}{\hat{\theta}} &  0\leq x\leq \hat{\theta}\\
        0 & \text{otherwise}
    \end{cases}
\]
می خواهیم $A = P[Xـ1 =x_1,X_2 = x_2,...,X_n= x_n|\hat{\theta}]$ را ماکسیمم کنیم که به خاطر استقلال داریم:\\
\[
    A =
    \begin{cases}
        0 &  max(X_i) > \hat{\theta}\\
        \frac{1}{{\hat{\theta}}^n} & \text{otherwise}
    \end{cases}
\]
پس داریم $max(X_i) \geq \hat{\theta}$ که چون تابع $\frac{1}{{\hat{\theta}}^n}$
بر حسب $\hat{\theta}$ نزولی است پس ماکسیمم خود را وقتی اخذ میکند که $\hat{\theta}$
کمترین مقدار خود یعنی $max(X_i)$ باشد.
\\\\
\subproblem{}
اگر بازه را باز در نظر بگیریم آنگاه نمیتوان $\hat{\theta} = max(X_i)$
را در نظر گرفت زیرا پارامتر ما از هر یک از $X_i$ ها بزرگتر اکید است.
حالا میتوانیم $\hat{\theta}$
را در بازه $(max(X_i),\infty)$ در نظر بگیریم 
که چون این بازه مینیمم ندارد روش بیشینه درست نمایی مقداری را به ما به عنوان
بر آوردگر اعلام نمیکند.
