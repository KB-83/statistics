\parte{}
داریم :
\[ (X-\mu)^{T} \Sigma^{-1} (X-\mu) = Y^{T} \Sigma^{-1} Y = \sum_{j = 1}^{n} {Y_j \sum_{i = 1}^{n} {Y_i a_{ij}}} =\]
 \[  \sum_{j = 2}^{n} {Y_j \sum_{i = 2}^{n} {Y_i a_{ij}}} + 
(Y_1)^2 a_{11}+{Y_1 \sum_{i = 2}^{n} {Y_i a_{i1}}}+ 
\sum_{j = 2}^{n} {Y_j {Y_1 a_{1j}}}\]
که باتوجه به تقارن $\Sigma^{-1}$ داریم
$a_{ij} = a_{ji}$ پس داریم:\\
\[  \sum_{j = 2}^{n} {Y_j \sum_{i = 2}^{n} {Y_i a_{ij}}} + 
(Y_1)^2 a_{11}+{Y_1 \sum_{i = 2}^{n} {Y_i a_{i1}}}+ 
Y_1 \sum_{i = 2}^{n} {Y_i { a_{i1}}}\]
\[ = 
a_{11}Y_1^2 +
{2 \sum_{i = 2}^{n} {Y_i a_{i1}}} Y_1+ 
\sum_{j = 2}^{n} {Y_j \sum_{i = 2}^{n} {Y_i a_{ij}}}
 \]
 درنیجه داریم :\\
\[ \alpha = a_{11} \]
\[\beta = \sum_{i = 2}^{n} {Y_i a_{i1}} \]
\[ \gamma = \sum_{j = 2}^{n} Y_j \sum_{i = 2}^{n} {Y_i a_{ij}}\]

\parte{}
داریم :\\
\[ \int_{-\infty}^{\infty}{f_{X}(x) dx_{1} = }\]

اگر قرار دهیم $A = \frac{1}{{(2\pi)^{n/2} \Sigma^{1/2}}} $ داریم:\\

\[ A \int_{-\infty}^{\infty}e^{-\frac{1}{2} \alpha Y_1^2 + 2\beta Y_1 + \gamma} dx_{1} = 
A \int_{-\infty}^{\infty}e^{-\frac{1}{2} (\alpha (Y_1^2 + \frac{\beta}{\alpha})^2 + \gamma - \frac{\beta^2}{\alpha})} dx_{1} \]
\\
\[ =
    A  e^{-\frac{1}{2}(\gamma - \frac{\beta^2}{\alpha})} \int_{-\infty}^{\infty}e^{-\frac{1}{2} (\alpha (Y_1^2 + \frac{\beta}{\alpha})^2)} dx_{1} = 
    A  e^{-\frac{1}{2}(\gamma - \frac{\beta^2}{\alpha})} \int_{-\infty}^{\infty}e^{-\frac{1}{2} u^2} \frac{1}{\sqrt{\alpha}} du = 
    A  e^{-\frac{1}{2}(\gamma - \frac{\beta^2}{\alpha})} \sqrt{2\pi} \frac{1}{\sqrt{\alpha}}
\]
\\
\[ = 
 \frac{(2\pi)^{\frac{1}{2}}}{\sqrt{\alpha} (2\pi)^{n/2} \Sigma^{1/2}} e^{-\frac{1}{2}(\gamma - \frac{\beta^2}{\alpha})} = 
 \frac{1}{\sqrt{\alpha} (2\pi)^{(n - 1)/2} (\Sigma)^{1/2}} e^{-\frac{1}{2}(\gamma - \frac{\beta^2}{\alpha})} 
\]




\parte{}
ابتدا ماتریس $B$  را نشان میدهم.\\
\[ B = \tilde{A} - \frac{vv^T}{a_{11}}\]
\[
    \tilde{A} = \begin{bmatrix}
    a_{22} & a_{23} & ... & a_{2n} \\
    a_{32} & a_{33} & ... & a_{3n} \\
    ... & ... & ... & ... \\
    a_{n2} & a_{n3} & ... & a_{nn} \\
\end{bmatrix}
\]


\[
    \frac{vv^T}{a_{11}} = \begin{bmatrix}
    \frac{a_{12}a_{12}}{a_{11}} & \frac{a_{12}a_{13}}{a_{11}} & ... &\frac{a_{12}a_{1n}}{a_{11}} \\
    \frac{a_{13}a_{12}}{a_{11}} & \frac{a_{13}a_{13}}{a_{11}} & ... & \frac{a_{13}a_{1n}}{a_{11}} \\
    ... & ... & ... & ... \\
    \frac{a_{1n}a_{12}}{a_{11}} & \frac{a_{1n}a_{13}}{a_{11}} & ... & \frac{a_{1n}a_{1n}}{a_{11}} \\
\end{bmatrix}
\]

\[
    B = \begin{bmatrix}
    a_{22} - \frac{a_{12}a_{12}}{a_{11}}  & a_{23} -\frac{a_{12}a_{13}}{a_{11}} & ... &a_{2n} - \frac{a_{12}a_{1n}}{a_{11}} \\
    a_{32} - \frac{a_{13}a_{12}}{a_{11}} & a_{33} - \frac{a_{13}a_{13}}{a_{11}} & ... & a_{3n} - \frac{a_{13}a_{1n}}{a_{11}} \\
    ... & ... & ... & ... \\
    a_{n2} - \frac{a_{1n}a_{12}}{a_{11}} & a_{n3} - \frac{a_{1n}a_{13}}{a_{11}} & ... & a_{nn} - \frac{a_{1n}a_{1n}}{a_{11}} \\
\end{bmatrix}
\]
 حال داریم :
\[ \tilde{Y}^{T}B\tilde{Y} = \sum_{j = 2}^{n}{Y_j \sum_{i = 2}^{n}{Y_i B_{ij}}} = 
\sum_{j = 2}^{n}{Y_j \sum_{i = 2}^{n}{Y_i (a_{ij} - \frac{a_{1i}a_{1j}}{a_{11}})}}
= \sum_{j = 2}^{n}{Y_j \sum_{i = 2}^{n}{Y_i (a_{ij})}}
- \sum_{j = 2}^{n}{Y_j \sum_{i = 2}^{n}{Y_i (\frac{a_{1i}a_{1j}}{a_{11}})}}
\]

\[ 
    = \gamma - \frac{1}{a_{11}}\sum_{j = 2}^{n}{Y_j a_{1j}\sum_{i = 2}^{n}{Y_i {a_{1i}}}} = 
    = \gamma - \frac{1}{a_{11}}(\sum_{i = 2}^{n}{Y_i a_{1i}})^2
    = \gamma - \frac{\beta ^2}{\alpha}
\]

\parte{}

می دانیم عملیات سطری مقدماتی روی یک ماتریس دترمینان آن را عوض نمیکند پس با استفاده از $a_{11}$ در $\Sigma^{-1}$ ستون اول این ماتریس را صفر میکنیم.
به وضوح سطر $i$ ام باید با $-\frac{a_{i1}}{a_{11}}$ برابر سطر اول جمع شود تا درآیه اول آن صفر شود.
که ماتریس به دست آمده به صورت زیر است \\

\[
    {\Sigma^{-1^{\prime}}} = \begin{bmatrix}
    a_{11} & a_{12} & ... & a_{1n} \\
    0      & a_{22} -  \frac{a_{12}a_{12}}{a_{11}}& ... & a_{2n} - \frac{a_{12}a_{1n}}{a_{11}}\\
    0      & a_{32} - \frac{a_{13}a_{12}}{a_{11}} & ... & a_{3n} - \frac{a_{13}a_{1n}}{a_{11}}\\
    ... & ... & ... & ... \\
    0      & a_{n2}- \frac{a_{1n}a_{12}}{a_{11}} & ... & a_{nn}- \frac{a_{1n}a_{1n}}{a_{11}} \\
\end{bmatrix}
\]
که اگر دقت کنیم به صورت زیر است
\[ 
    {\Sigma^{-1^{\prime}}} = \begin{bmatrix}
        a_{11} & v^{T} \\
        0 & B \\
    \end{bmatrix}
\]
پس داریم :\\
\[ det(\Sigma^{-1}) = det \begin{bmatrix}
    a_{11} & v^{T} \\
    0 & B \\
\end{bmatrix}
\]
که همان خواسته سوال است.\\












\parte{}
داریم :\\
\[
    \Sigma \Sigma^{-1} = I
\]\\
\[
    \begin{bmatrix}
        a_{11} & v^{T} \\
        v & \tilde{A} \\
    \end{bmatrix}
    \begin{bmatrix}
        \sigma_{11} & w^{T} \\
        w & \tilde{\Sigma} \\
    \end{bmatrix}
    =I
\]\\
پس داریم:\\
\[
    a_{11}\sigma_{11}+v^T w = 1
    => w = (1 - a_{11}\sigma_{11})
\]

\[
    a_{11}w^T+v^T \tilde{\Sigma} = 0 = (0,0,0,...,0)_{1\times(n-1)}
\]

\[
    v\sigma_{11}+\tilde{A}w = 0 = (0,0,0,...,0)^{T}_{1\times(n-1)}
\]

\[
    vw^T+\tilde{A}\tilde{\Sigma} = I_{(n-1)\times(n-1)}
\]
می خواهیم نشان دهیم$B\tilde{\Sigma} = I$.داریم:\\
\[
    (\tilde{A}-\frac{vv^T}{a_{11}})\tilde{\Sigma}  
    =\tilde{A}\tilde{\Sigma} -\frac{vv^T\tilde{\Sigma}}{a_{11}}
\]\\
حالا با توجه به رابطه دوم میدانیم $v^T\tilde{\Sigma} = -a_{11}w^T$
آن را در عبارت بالا جایگزین میکنیم داریم:\\
\[
    =>(\tilde{A}-\frac{vv^T}{a_{11}})\tilde{\Sigma}  
    =\tilde{A}\tilde{\Sigma} -\frac{-a_{11}vw^T}{a_{11}}
    =\tilde{A}\tilde{\Sigma} +vw^T
\]
که همان عبارت چهارم و برابر با $I$ است پس داریم:\\
\[
    \tilde{A}\tilde{\Sigma} +vw^T = B\tilde{\Sigma} = I
\]
که ادعا ثابت میشود.\\\\