\documentclass[12pt]{article}
\usepackage{/Users/kajal/Documents/statistics/resources/templates/HomeWorkTemplate}
\usepackage{circuitikz}
\usepackage[shortlabels]{enumitem}
\usepackage{hyperref}
\usepackage{tikz}
\usepackage{fontspec}
\usepackage{xepersian}
\usepackage{graphicx}
\usepackage{float}
\usepackage{changepage}

\usetikzlibrary{arrows,automata}
\usetikzlibrary{circuits.logic.US}
\settextfont{XB Niloofar}

\newcounter{problemcounter}
\newcounter{subproblemcounter}[problemcounter]
\newcounter{partcounter}[subproblemcounter]

\renewcommand{\thesubproblemcounter}{\alph{subproblemcounter})}
\renewcommand{\thepartcounter}{\roman{partcounter})}

\newcommand{\problem}[1]{
    \setcounter{subproblemcounter}{0}
    \stepcounter{problemcounter}
    \subsection*{پرسش \arabic{problemcounter} #1}
}

\newcommand{\subproblem}[1]{
    \setcounter{partcounter}{0}
    \stepcounter{subproblemcounter}
    \textbf{\harfi{subproblemcounter})}
}

\newcommand{\parte}[1]{
    \stepcounter{partcounter}
	\arabic{partcounter})
}



\begin{document}

\handout
{آمار و کاربرد ها}
{تمرین سری سه}
{کژال باغستانی}
{۴۰۱۱۰۰۰۷۱}

% کامنت بخش ث:\\
% از عارف بپرس چرا دی ایکس با دی وای برابر میشه چرا سیگما تانیر گذار نیست؟\\
\problem{}
\subproblem{}
\parte{}

\parte{}
در نظر میگیریم:
$Y = \sqrt{\Sigma ^{-1}}(X-\mu)$
از آنجایی که $\Sigma$ متقارن پس است پس $\Sigma^{-1}$ نیز متقارن 
در نتیجه $\sqrt{\Sigma^{-1}}$ نیز متقارن است پس داریم:\\
\[\sqrt{\Sigma^{-1}} = \sqrt{\Sigma^{-1}}^{T} \]
پس داریم:\\
\[ Y^{T}Y = (\sqrt{\Sigma ^{-1}}(X-\mu))^{T}\sqrt{\Sigma ^{-1}}(X-\mu) = (X-\mu)^{T}\Sigma^{-1}(X-\mu) \]
% (X-\mu)^{T}\sqrt{\Sigma ^{-1}}\sqrt{\Sigma ^{-1}}(X-\mu)
\parte{}
در نظر میگیریم:
$A = \frac{1}{{(2\pi)^{n/2} |\Sigma|^{1/2}}} $
میخواهیم انتگرال زیر را محاسبه کنیم:\\
\[
A \int_{\mathbb{R}^{n}} e^{-\frac{1}{2} (\mathbf{x}-\boldsymbol{\mu})^T \Sigma^{-1} (\mathbf{x}-\boldsymbol{\mu})} \, d{X}
\]
در نظر میگیریم : 
$Y = \sqrt{\Sigma ^{-1}}(X-\mu)$ حالا به محاسبه ژاکوبین $Y$  می پردازیم:\\

\[D(Y(X)) = \sqrt{\Sigma^{-1}}\]
\[J_{Y} = det(\sqrt{\Sigma^{-1}}) =\sqrt{det(\Sigma^{-1})} = \frac{1}{\sqrt{det(\Sigma)}}ke in ghalate dorostesh kon  \]
% میدانیم $\Sigma = P^{T}DP$ پس:
% \[det(\Sigma) = det(P^{T}DP) = det(P^{T})det(D)det(P) = det(P^{-1})det(P)det(D) = det(D)\]
% پس داریم:
% $\sqrt{det(\Sigma)} = \sqrt{det(D)}$ و از طرف دیگر مشابهه قسمت قبل
% $det(\sqrt{\Sigma}) = det(\sqrt{D})$ و از آنجایی که ماتریس $D$ ماتریسی قطری است داریم:
% $\sqrt{det(D)} = det(\sqrt{D})$ پس تساوی قسمت قبل را کامل میکنیم و داریم: \\
% \[J_{Y} = det(\sqrt{\Sigma^{-1}}) = det(\sqrt{\Sigma}) = {det(\Sigma)}^{\frac{1}{2}} \]
پس در ادامه با تغییر متغیر در انتگرال داریم:\\
\[A \int_{\mathbb{R}^{n}} e^{-\frac{1}{2} Y^{T}Y} {det(\Sigma)}^{\frac{1}{2}}\, d{Y} = \frac{1}{(2\pi)^{\frac{n}{2}}}\int_{\mathbb{R}^{n}} e^{-\frac{1}{2} Y^{T}Y} \, d{Y}\]
که داریم :\\
\[ \frac{1}{(2\pi)^{\frac{n}{2}}}\int_{\mathbb{R}^{n}} e^{\sum_{i = 1}^{n}{-\frac{1}{2}Y_{i}^2} } \, d{Y} \]
و از انجایی که میدانیم $\int_{\mathbb{R}}{e^{-\frac{1}{2}Y_{i}^2}}d{Y_{i}} = (2\pi)^{\frac{1}{2}}$ در نتیجه داریم:
\[ \int_{\mathbb{R}^{n}} e^{\sum_{i = 1}^{n}{-\frac{1}{2}Y_{i}^2} } \, d{Y} = (2\pi)^{\frac{n}{2}}\] پس :\\

\[ \frac{1}{(2\pi)^{\frac{n}{2}}}\int_{\mathbb{R}^{n}} e^{\sum_{i = 1}^{n}{-\frac{1}{2}Y_{i}^2} } \, d{Y}  = \frac{(2\pi)^{\frac{n}{2}}}{(2\pi)^{\frac{n}{2}}} = 1\]


\subproblem{}
\parte{}
از آنجایی که $X_i$ 
ها هم توزیع هستند $P[a<X_i<b] = P[a<X_j<b]$
در نتیجه اگر بخواهیم $X_{(1)}$ بین $a$ و $b$
باشد هر یک از $X_i$ ها میتوانند کاندید باشند پس
در کل برای اینکه ترتیب $X_i$ ها را بچینیم $n!$
حالت داریم که این یعنی :\\
\[
  Pr[x_1-\frac{\epsilon}{2}<X_{(1)}<x_1+\frac{\epsilon}{2},
  ...,
  x_n-\frac{\epsilon}{2}<X_{(n)}<x_n+\frac{\epsilon}{2}]  
\]
\[
  = n!Pr[x_{i_1}-\frac{\epsilon}{2}<X_{1}<x_{i_1}+\frac{\epsilon}{2},
  ...,
  x_{i_n}-\frac{\epsilon}{2}<X_{n}<x_{i_n}+\frac{\epsilon}{2}]  
\]
حالا با توجه به مستقل بودن $X_i$ ها داریم:\\

\[
  = n!Pr[x_{i_1}-\frac{\epsilon}{2}<X_{1}<x_{i_1}+\frac{\epsilon}{2},
  ...,
  x_{i_n}-\frac{\epsilon}{2}<X_{n}<x_{i_n}+\frac{\epsilon}{2}] 
  = n! \prod_{j=1}^{n}{Pr[x_{i_j} - \frac{\epsilon}{2} <X_j< x_{i_j} + \frac{\epsilon}{2}]}
\]
که عبارت $Pr[x_{i_j} - \frac{\epsilon}{2} <X_j< x_{i_j} + \frac{\epsilon}{2}]$
وقتی $\epsilon$
به سمت صفر میرود به سمت $\epsilon f(x_j)$ می رود
پس داریم:\\
\[
    \approx n!\epsilon^n f(x_1)f(x_2)...f(x_n)
\]
\parte{}
با توجه به تعریف داریم :\\
\[
    Pr[x_1-\frac{\epsilon}{2}<X_{(1)}<x_1+\frac{\epsilon}{2},
  ...,
  x_n-\frac{\epsilon}{2}<X_{(n)}<x_n+\frac{\epsilon}{2}]
\]
\[
    = \int_{x_n-\frac{\epsilon}{2}}^{x_n+\frac{\epsilon}{2}} 
    \dots
    \int_{x_1-\frac{\epsilon}{2}}^{x_1+\frac{\epsilon}{2}} 
    {f_{X_{(1)},...,X_{(n)}}(x_1,x_2,..,x_n)}
    dx_1 \dots dx_n
\]
که داریم:\\
\[
    \lim_{\epsilon \to 0}{
    \int_{x_n-\frac{\epsilon}{2}}^{x_n+\frac{\epsilon}{2}} 
    \dots
    \int_{x_1-\frac{\epsilon}{2}}^{x_1+\frac{\epsilon}{2}} 
    {f_{X_{(1)},...,X_{(n)}}(x_1,x_2,..,x_n)}
    dx_1 \dots dx_n
    }
    = \epsilon^n{f_{X_{(1)},...,X_{(n)}}(x_1,x_2,..,x_n)}
\]
پس به صورت تقریبی داریم:\\
\[
    {f_{X_{(1)},...,X_{(n)}}(x_1,x_2,..,x_n)} = n!f(x_1)f(x_2)...f(x_n)
\]
\subproblem{}
\parte{}
اگر توزیع حاشیه ای نسبت به متغیر $X_i$  را داشته باشیم میتوانیم امید ریاضی آن را محاسبه کنیم و اما با توجه به نتیجه ای که قسمت قبل گرفتیم
توزیع حاشیه ای نسبت به دو متغیر $X_i$ و $X_j$ برابر است با:\\
\[ 
    \frac{1}{\sqrt{(2\pi)^2 det(\tilde{\Sigma})}} \text{exp}(-\frac{1}{2}{(\tilde{X} - \tilde{\mu})}^{T}\Tilde{\Sigma}^{-1}(\tilde{X} - \tilde{\mu}))
\]
که $\tilde{X} = (X_i,X_j)$ و $\tilde{\mu} = (\mu_i,\mu_j)$ و $\Sigma = 
\begin{bmatrix}
    \sigma_{ii} & \sigma_{ij} \\
    \sigma_{ij} & \sigma_{jj} \\
\end{bmatrix}
$.\\
حال اگر نسبت به $X_j$ از این تابع انتگرال بگیریم دوباره با توجه به قسمت قبل تابع توزیع حاشیه ای برای $X_i$ را خواهیم داشت
فقط به این نکته توجه کنیم که $\tilde{\Sigma} = det(\tilde{\Sigma}) = \sigma_{ii}$ پس داریم:\\
\[ 
    f_{X_i}(x_i) =  \frac{1}{\sqrt{(2\pi)\sigma_{ii}}} \text{exp}(-\frac{(X_i - \mu_i)^2}{2\sigma_{ii}})
\]
\\
 که میدانیم تابع چگالی متغیر تصادفی نرمال با میانگین $\mu_i$ است پس:
 \[ E[X_i] = \mu_i\]
\parte{}
طبق استدلال قسمت قبل و تابع چگالی بدست آمده به وضوح داریم:\\
\[\sigma_{ii} = \sigma_{i}^2 = var(X_i) \]
\parte{}
طبق قسمت $1$ داریم:\\
\[
    f_{X_i,X_j}(x_i,x_j)=\frac{1}{\sqrt{(2\pi)^2 det(\tilde{\Sigma})}} \text{exp}(-\frac{1}{2}{(\tilde{X} - \tilde{\mu})}^{T}\Tilde{\Sigma}^{-1}(\tilde{X} - \tilde{\mu}))
\]
\subproblem{}
\parte{}
تغییر متغیر $u = G(x)$ را انجام میدهیم داریم:\\
\[
    u = G(x)
    \quad
    =>
    \quad
    \frac{du}{dx} = g(x)
    \quad
    =>
    \quad
    dx = \frac{du}{g(x)}
\]
حالا عبارات بدست آمده را در انتگرال جایگذاری میکنیم داریم:\\
\[
    var(X_{(n)}) = \int_{-\infty}^{\infty}{
    \alpha_n {H(u)}^2(u(1-u))^{n-1} du
  }
\]

\parte{}
\[
    H(u) = H(\frac{1}{2})+{H^{\prime}(\frac{1}{2})}(u-\frac{1}{2}) 
\]
\parte{}
 قرار می دهیم $H(\frac{1}{2}) = a$و $H(\frac{1}{2}) = b$.

\[
    var(X_{(n)}) = \alpha_n  \int_{-\infty}^{\infty}{
    {(a+b(u-\frac{1}{2}))}^2(u(1-u))^{n-1} du
  }
\]
\[
    = \alpha_n  \int_{-\infty}^{\infty}{
    {(a+b(\frac{z}{2}))}^2(\frac{z+1}{2}(1-\frac{z+1}{2}))^{n-1} \frac{dz}{2}
  }
\]

\[
    = \alpha_n  \int_{-\infty}^{\infty}{
    (a^2+\frac{b^2z^2}{4}+abz)
    (\frac{z+1}{2} - \frac{z^2+2z+1}{4})^{n-1}
     \frac{dz}{2}
  }
\]
\[
    = \alpha_n  \int_{-\infty}^{\infty}{
    (a^2+\frac{b^2y}{4}+aby^\frac{1}{2})
    (\frac{y^\frac{1}{2}+1}{2} - \frac{y+2y^\frac{1}{2}+1}{4})^{n-1}
     \frac{dy}{4z}
  }
\]

% \[
%     = \alpha_n  \int_{-\infty}^{\infty}{
%     (a^2 + b^2(u^2 + \frac{1}{4} - u)+2ab(u-\frac{1}{2}))
%     (u(1-u))^{n-1} du
%   }
% \]
\subproblem{}
\parte{}
\[M_X(t) = E[e^{t^{T}X}] = \int_{\mathbb{R}^{n}}{f_{X}(x) e^{t^{T}X}} dX \]
که با توجه به تعریف $F_X(x)$ داریم:
\[M_X(t) =  \int_{\mathbb{R}^{n}}{\frac{1}{\sqrt{(2\pi)^n det(\Sigma)}} e^{-\frac{1}{2}(X-\mu)^{T}\Sigma^{-1}(X-\mu)} e^{t^{T}X}} dX\]
\[ = \int_{\mathbb{R}^{n}}{\frac{1}{\sqrt{(2\pi)^n det(\Sigma)}} e^{t^{T}X-\frac{1}{2}(X-\mu)^{T}\Sigma^{-1}(X-\mu)}} dX \]
که همان خواسته سوال است.\\

\parte{}
\[ -\frac{1}{2}(X-\mu-\Sigma t)^{T}\Sigma^{-1}(X-\mu-\Sigma t) 
= -\frac{1}{2}((X-\mu)^{T}\Sigma^{-1}-t^{T}\Sigma^{T}\Sigma^{-1})(X-\mu-\Sigma t) \]
\[ 
= -\frac{1}{2}((X-\mu)^{T}\Sigma^{-1}-t^{T})((X-\mu)-\Sigma t) 
\]
\[ 
= -\frac{1}{2}((X-\mu)^{T}\Sigma^{-1}(X-\mu) -t^{T}(X-\mu) - (X-\mu)^{T}t +t^{T}\Sigma t)
\]
\[
= -\frac{1}{2}(X-\mu)^{T}\Sigma^{-1}(X-\mu) + t^{T}(X-\mu) -\frac{1}{2}t^{T}\Sigma t
\]\\

\parte{}
می خواهیم انتگرال زیر را محاسبه کنیم:
\[ \int_{\mathbb{R}^{n}}{\frac{1}{\sqrt{(2\pi)^n det(\Sigma)}} e^{t^{T}X-\frac{1}{2}(X-\mu)^{T}\Sigma^{-1}(X-\mu)}} dX \]

با توجه به رابطه بالا داریم:\\

\[
    -\frac{1}{2}(X-\mu)^{T}\Sigma^{-1}(X-\mu) + t^{T}(X-\mu) -\frac{1}{2}t^{T}\Sigma t
    +(\frac{1}{2}t^{T}\Sigma t + t^{T}\mu)
\]

\[
= -\frac{1}{2}Y^{T}{\Sigma}^{-1}Y +(\frac{1}{2}t^{T}\Sigma t + t^{T}\mu)
= t^{T}X-\frac{1}{2}(X-\mu)^{T}\Sigma^{-1}(X-\mu)
\]\\

حالا با تغییر متغیر $Y=X-\mu-\Sigma t$ داریم:\\

\[ 
    D(Y) = I \quad J(D(Y)) = det(I) = 1
\]\\

\[ 
    \int_{\mathbb{R}^{n}}{\frac{1}{\sqrt{(2\pi)^n det(\Sigma)}} e^{-\frac{1}{2}Y^{T}{\Sigma}^{-1}Y +(\frac{1}{2}t^{T}\Sigma t + t^{T}\mu)}} dY 
\]\\

\[
    = e^{(\frac{1}{2}t^{T}\Sigma t + t^{T}\mu)}\int_{\mathbb{R}^{n}}{\frac{1}{\sqrt{(2\pi)^n det(\Sigma)}} e^{-\frac{1}{2}Y^{T}{\Sigma}^{-1}Y }} dY 
\]\\
و از طرفی میدانیم:\\
\[ \int_{\mathbb{R}^{n}}{\frac{1}{\sqrt{(2\pi)^n det(\Sigma)}} e^{-\frac{1}{2}Y^{T}{\Sigma}^{-1}Y }} dY  = 1\]
پس:\\
\[ M_X(t) = \text{exp}[\frac{1}{2}t^{T}\Sigma t + t^{T}\mu]\]


نماد گذاری ژاکوبین و دی رو درست کن و از عارف بپرس چرا دی ایکس با دی وای برابر میشه چرا سیگما تانیر گذار نیست؟
\subproblem{}
تابع مولد گشتاور را برای متغیر تصادفی Y
محاسبه میکنیم. داریم:\\
\[
    M_Y(t) = E[e^{t^{T} Y}]
    = E[e^{t^{T} (c+BX)}]
    = e^{t^{T}c}E[e^{t^{T}BX}]
\]\\
حالا تغییر متغیر $t^\prime = t^{T}B$
را درنظر میگیریم و چون $X$ متغیر تصادفی نرمال با
پارامتر های $\mu$ و $\Sigma$ است
داریم:\\
\[
    E[e^{t^\prime X}]
    = e^{\frac{1}{2}{t^\prime}\Sigma{t^\prime}^{T}+{t^\prime}\mu}
    = e^{\frac{1}{2}{t^{T}B}\Sigma{t{B}^{T}}+{t^{T}B}\mu}
\]\\
پس داریم :\\
\[
    M_Y(t) = e^{t^{T}c}e^{\frac{1}{2}{t^{T}B}\Sigma{{B}^{T}t}+{t^{T}B}\mu}
    = e^{\frac{1}{2}{t^{T}B}\Sigma{{B}^{T}t}+t^{T}(c+B\mu)}
\]\\
فرض کنیم $A = B\Sigma B^{T}$ آنگاه داریم:\\
\[
    t^{T}At = t^ B\Sigma B^{T}t 
\]\\
پس داریم:\\
\[
    M_Y(t) = e^{\frac{1}{2}{t^{T}At}+t^{T}(c+B\mu)}
\]\\
که از روی تعریف تابع مولد گشتاور برای توزیع نرمال داریم:\\
\[
    Y \sim \mathcal{N}(c+B\mu, A) 
    => Y \sim \mathcal{N}(c+B\mu, B\varSigma B^T)
\]

\problem{}
\subproblem{}
قانون قوی اعداد بزرگ میگوید:\\
\[ \lim_{n \to \infty} P(\frac{X1+X2+X3+...+Xn}{n} = \mu) = 1\]
یا به بیان دیگر 
\[ \bar{X}_n \to \mu \quad a.s. \quad as \quad n\to\infty \]
\\
و یکی از مثال های جالب آن این است که در یک بازی که احتمال برد قمار باز $p$ است و احتمال برد کازینو \(1-p\) که $p<\frac{1}{2}$
پس از بازی های زیاد هرچند شاید دفعه اول قمار باز برنده باشد ولی در نهایت کازینو برنده است زیرا تعداد برد های کازینو نسبت به تعداد کل بازی ها به سمت \(1-p\) می رود که
بیشتر از $\frac{1}{2}$ است.\\

\subproblem{}
تفاوت صورت قوی و ضعیف قانون اعداد بزرگ این است که در صورت قوی آن مجموعه پشامد هایی
که در قانون صدق نمیکنند نسبت به مجموعه کل پشامد ها اندازه صفر است و این پیشامد ها ثابت و مشخص هستند
در صورتی که در صورت ضیف آن مجموعه این پیشامد ها ثابت و مشخص نیستند و این قانون میگوید صرفا با زیاد کردن تعداد
نمونه ها نسبت این پیشامد ها به مجموعه کل پیشامد ها کم و کم تر میشود اما درباره اینکه این پیشامد ها 
ثابت هستند تضمینی نمیدهد.\\\\
برهان:\\
می خواهیم نشان دهیم :\\ 
\[ \forall \epsilon >0 \quad \quad \lim_{{n \to \infty}} P(\lvert \bar{X} - \mu \rvert > \epsilon) = 0 \]
با توجه به نامساوی مارکوف داریم\\
\[P(X>a)\leq \frac{E[X]}{a} \] 
پس داریم
\[P((\bar{X} - \mu)^2>a^2)\leq\frac{E[( \bar{X} - \mu)^2]}{a^2} \] 
که این یعنی
\[P((\bar{X} - \mu)^2>a^2)\leq \frac{var(\bar{X})}{a^2} \] 
و با توجه به $i.i.d$ بودن $Xi$ ها 
\[var(\bar{X}) = \frac{\sigma^2}{n}\]
\[ P(\lvert \bar{X} - \mu \rvert > \epsilon ) \leq \frac{\sigma^2}{n\epsilon} \quad for \quad fixed \quad \epsilon \quad and \quad  \sigma<\infty\] 
پس:
\[0 \leq \lim_{{n \to \infty}} P(\lvert \bar{X} - \mu \rvert > \epsilon) \leq \lim_{{n \to \infty}} \frac{\sigma^2}{n\epsilon} = 0 \]
که ثابت میشود:
\[ \forall \epsilon >0 \quad \quad \lim_{{n \to \infty}} P(\lvert \bar{X} - \mu \rvert > \epsilon) = 0 \]

\subproblem{}
کد پایتون:\\
\begin{figure}[H]
	\centering
	\includegraphics[width=0.5\textwidth]{/Users/kajal/Documents/statistics/resources/hw2/Figure_3.png}
\end{figure}

خروجی کد:
\begin{figure}[H]
	\centering
	\includegraphics[width=0.5\textwidth]{/Users/kajal/Documents/statistics/resources/hw2/Figure_4.png}
\end{figure}

\problem{}

\subproblem{}
\begin{figure}[H]
	\centering
	\includegraphics[width=0.5\textwidth]{/Users/kajal/Documents/statistics/resources/hw1/Figure_1.png}
	\caption{نمودار توزیع تجمعی رسم شده توسط python}
\end{figure}

\subproblem{}
برای به دست اوردن میانگین کافیست مقدار مرکز دسته ها را محاسبه کنیم و برای هر دسته مقدار مرکز دسته را در فراوانی نسبی دسته
ضرب کنیم و در نهایت با هم جمع کنیم.
\newline
\newline
\[ \bar{X} = \frac{4 * 167.5 + 5 * 172.5 + 3 * 177.5 + 7 * 182.5 + 5 * 187.5 + 2 * 192.5}{26} = 179.423 \]
\newline
\newline
حالا برای پیدا کردن میانه کافیست با ببینیم در نمودار کدام دسته است که دسته قبل از آن کمتر از $13$ و دسته بعد از آن بیشتر از $13$ فراوانی تجمعی دارد.
که با توجه به شکل دسته $180$ تا $185$ دسته مورد نظر ماست که ابتدای دسته $12$ و انتهای آن $19$ است و با قضیه تالس به سادگی مشخص میشود که اگر یک خط از نقطه $(180,12)$ و $(185,19)$ رد کنیم
در نقطه $180 + \frac{5}{7}$ یعنی تقریبا $180.71 $میانه ما به دست می آید.
که نسبت تالس به صورت زیر است:\newline
\begin{figure}[H]
	\centering
	\includegraphics[width=0.5\textwidth]{/Users/kajal/Documents/statistics/resources/hw1/Figure_4.png}
	\caption{پیدا کردن میانه در نمودار هیستوگرام به کمک روش تالس}
\end{figure}

\subproblem{}
اعداد به دست آمده توسط پایتون به صورت زیر هستند که دقیقا با رابطه گفته شده در کلاس
یکی هستند.

\[ Mean Root for -1: 178.201273382513\]
\[Geometric Mean: 178.37000921518623\]
\[Mean Root for 1: 178.53846153846152\]
\[Mean Root for 2: 178.7064632295094\]

\problem{}

\subproblem{}
کافیست یک مثال نقض از توزیعی بیاورم که میانگین آن (که برای توزیع پیوسته به صورت امید ریاضی تعریف میشود) جایی باشد که تابع توزیع آن برابر با یک دوم نشود. برای مثال در شکل زیر تنها نقطه ای که F(x) برای آن برابر با یک دوم میشود صفر است اما به وضوح امید ریاضی یا همان میانگین این نمودار بزرگ تر از صفر است.

\begin{figure}[H]
	\centering
	\includegraphics[width=0.5\textwidth]{/Users/kajal/Documents/statistics/resources/hw1/Figure_5.png}
	\caption{مثال نقض برای قسمت آ سوال چهار}
\end{figure}

\subproblem{}
اگر میانه توزیع را برای حالت پیوسته تعریف کنیم نقطه ای در تابع چگالی که در آن انتگرال قبل و بعد از أن برابر با یک دوم میشود این دقیقا تعریفی هست که در صورت سوال گفته شده و میانه در آن می افتد.

\subproblem{}
برای این قسمت کافیست یک توزیع نامتقارن ارایه دهیم که هر دوی میانه و میانگین عوض این مجموعه باشند.
\begin{figure}[H]
	\centering
	\includegraphics[width=0.5\textwidth]{/Users/kajal/Documents/statistics/resources/hw1/Figure_6.png}
\end{figure}
در شکل بالا به صورت شهودی مشخص است که میانگین و میانه هر دو در قسمت وسط نمودار می افتند و توزیع دقیقا متقارن نیست.


\end{document}

