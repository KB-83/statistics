\parte{}
\[M_X(t) = E[e^{t^{T}X}] = \int_{\mathbb{R}^{n}}{f_{X}(x) e^{t^{T}X}} dX \]
که با توجه به تعریف $F_X(x)$ داریم:
\[M_X(t) =  \int_{\mathbb{R}^{n}}{\frac{1}{\sqrt{(2\pi)^n det(\Sigma)}} e^{-\frac{1}{2}(X-\mu)^{T}\Sigma^{-1}(X-\mu)} e^{t^{T}X}} dX\]
\[ = \int_{\mathbb{R}^{n}}{\frac{1}{\sqrt{(2\pi)^n det(\Sigma)}} e^{t^{T}X-\frac{1}{2}(X-\mu)^{T}\Sigma^{-1}(X-\mu)}} dX \]
که همان خواسته سوال است.\\

\parte{}
\[ -\frac{1}{2}(X-\mu-\Sigma t)^{T}\Sigma^{-1}(X-\mu-\Sigma t) 
= -\frac{1}{2}((X-\mu)^{T}\Sigma^{-1}-t^{T}\Sigma^{T}\Sigma^{-1})(X-\mu-\Sigma t) \]
\[ 
= -\frac{1}{2}((X-\mu)^{T}\Sigma^{-1}-t^{T})((X-\mu)-\Sigma t) 
\]
\[ 
= -\frac{1}{2}((X-\mu)^{T}\Sigma^{-1}(X-\mu) -t^{T}(X-\mu) - (X-\mu)^{T}t +t^{T}\Sigma t)
\]
\[
= -\frac{1}{2}(X-\mu)^{T}\Sigma^{-1}(X-\mu) + t^{T}(X-\mu) -\frac{1}{2}t^{T}\Sigma t
\]\\

\parte{}
می خواهیم انتگرال زیر را محاسبه کنیم:
\[ \int_{\mathbb{R}^{n}}{\frac{1}{\sqrt{(2\pi)^n det(\Sigma)}} e^{t^{T}X-\frac{1}{2}(X-\mu)^{T}\Sigma^{-1}(X-\mu)}} dX \]

با توجه به رابطه بالا داریم:\\

\[
    -\frac{1}{2}(X-\mu)^{T}\Sigma^{-1}(X-\mu) + t^{T}(X-\mu) -\frac{1}{2}t^{T}\Sigma t
    +(\frac{1}{2}t^{T}\Sigma t + t^{T}\mu)
\]

\[
= -\frac{1}{2}Y^{T}{\Sigma}^{-1}Y +(\frac{1}{2}t^{T}\Sigma t + t^{T}\mu)
= t^{T}X-\frac{1}{2}(X-\mu)^{T}\Sigma^{-1}(X-\mu)
\]\\

حالا با تغییر متغیر $Y=X-\mu-\Sigma t$ داریم:\\

\[ 
    D(Y) = I \quad J(D(Y)) = det(I) = 1
\]\\

\[ 
    \int_{\mathbb{R}^{n}}{\frac{1}{\sqrt{(2\pi)^n det(\Sigma)}} e^{-\frac{1}{2}Y^{T}{\Sigma}^{-1}Y +(\frac{1}{2}t^{T}\Sigma t + t^{T}\mu)}} dY 
\]\\

\[
    = e^{(\frac{1}{2}t^{T}\Sigma t + t^{T}\mu)}\int_{\mathbb{R}^{n}}{\frac{1}{\sqrt{(2\pi)^n det(\Sigma)}} e^{-\frac{1}{2}Y^{T}{\Sigma}^{-1}Y }} dY 
\]\\
و از طرفی میدانیم:\\
\[ \int_{\mathbb{R}^{n}}{\frac{1}{\sqrt{(2\pi)^n det(\Sigma)}} e^{-\frac{1}{2}Y^{T}{\Sigma}^{-1}Y }} dY  = 1\]
پس:\\
\[ M_X(t) = \text{exp}[\frac{1}{2}t^{T}\Sigma t + t^{T}\mu]\]


نماد گذاری ژاکوبین و دی رو درست کن و از عارف بپرس چرا دی ایکس با دی وای برابر میشه چرا سیگما تانیر گذار نیست؟