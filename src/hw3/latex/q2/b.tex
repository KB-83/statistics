\parte{}
از آنجایی که $X_i$ 
ها هم توزیع هستند $P[a<X_i<b] = P[a<X_j<b]$
در نتیجه اگر بخواهیم $X_{(1)}$ بین $a$ و $b$
باشد هر یک از $X_i$ ها میتوانند کاندید باشند پس
در کل برای اینکه ترتیب $X_i$ ها را بچینیم $n!$
حالت داریم که این یعنی :\\
\[
  Pr[x_1-\frac{\epsilon}{2}<X_{(1)}<x_1+\frac{\epsilon}{2},
  ...,
  x_n-\frac{\epsilon}{2}<X_{(n)}<x_n+\frac{\epsilon}{2}]  
\]
\[
  = n!Pr[x_{i_1}-\frac{\epsilon}{2}<X_{1}<x_{i_1}+\frac{\epsilon}{2},
  ...,
  x_{i_n}-\frac{\epsilon}{2}<X_{n}<x_{i_n}+\frac{\epsilon}{2}]  
\]
حالا با توجه به مستقل بودن $X_i$ ها داریم:\\

\[
  = n!Pr[x_{i_1}-\frac{\epsilon}{2}<X_{1}<x_{i_1}+\frac{\epsilon}{2},
  ...,
  x_{i_n}-\frac{\epsilon}{2}<X_{n}<x_{i_n}+\frac{\epsilon}{2}] 
  = n! \prod_{j=1}^{n}{Pr[x_{i_j} - \frac{\epsilon}{2} <X_j< x_{i_j} + \frac{\epsilon}{2}]}
\]
که عبارت $Pr[x_{i_j} - \frac{\epsilon}{2} <X_j< x_{i_j} + \frac{\epsilon}{2}]$
وقتی $\epsilon$
به سمت صفر میرود به سمت $\epsilon f(x_j)$ می رود
پس داریم:\\
\[
    \approx n!\epsilon^n f(x_1)f(x_2)...f(x_n)
\]
\parte{}
با توجه به تعریف داریم :\\
\[
    Pr[x_1-\frac{\epsilon}{2}<X_{(1)}<x_1+\frac{\epsilon}{2},
  ...,
  x_n-\frac{\epsilon}{2}<X_{(n)}<x_n+\frac{\epsilon}{2}]
\]
\[
    = \int_{x_n-\frac{\epsilon}{2}}^{x_n+\frac{\epsilon}{2}} 
    \dots
    \int_{x_1-\frac{\epsilon}{2}}^{x_1+\frac{\epsilon}{2}} 
    {f_{X_{(1)},...,X_{(n)}}(x_1,x_2,..,x_n)}
    dx_1 \dots dx_n
\]
که داریم:\\
\[
    \lim_{\epsilon \to 0}{
    \int_{x_n-\frac{\epsilon}{2}}^{x_n+\frac{\epsilon}{2}} 
    \dots
    \int_{x_1-\frac{\epsilon}{2}}^{x_1+\frac{\epsilon}{2}} 
    {f_{X_{(1)},...,X_{(n)}}(x_1,x_2,..,x_n)}
    dx_1 \dots dx_n
    }
    = \epsilon^n{f_{X_{(1)},...,X_{(n)}}(x_1,x_2,..,x_n)}
\]
پس به صورت تقریبی داریم:\\
\[
    {f_{X_{(1)},...,X_{(n)}}(x_1,x_2,..,x_n)} = n!f(x_1)f(x_2)...f(x_n)
\]