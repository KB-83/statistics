\parte{}

\parte{}
در نظر میگیریم:
$Y = \sqrt{\Sigma ^{-1}}(X-\mu)$
از آنجایی که $\Sigma$ متقارن پس است پس $\Sigma^{-1}$ نیز متقارن 
در نتیجه $\sqrt{\Sigma^{-1}}$ نیز متقارن است پس داریم:\\
\[\sqrt{\Sigma^{-1}} = \sqrt{\Sigma^{-1}}^{T} \]
پس داریم:\\
\[ Y^{T}Y = (\sqrt{\Sigma ^{-1}}(X-\mu))^{T}\sqrt{\Sigma ^{-1}}(X-\mu) = (X-\mu)^{T}\Sigma^{-1}(X-\mu) \]
% (X-\mu)^{T}\sqrt{\Sigma ^{-1}}\sqrt{\Sigma ^{-1}}(X-\mu)
\parte{}
در نظر میگیریم:
$A = \frac{1}{{(2\pi)^{n/2} |\Sigma|^{1/2}}} $
میخواهیم انتگرال زیر را محاسبه کنیم:\\
\[
A \int_{\mathbb{R}^{n}} e^{-\frac{1}{2} (\mathbf{x}-\boldsymbol{\mu})^T \Sigma^{-1} (\mathbf{x}-\boldsymbol{\mu})} \, d{X}
\]
در نظر میگیریم : 
$Y = \sqrt{\Sigma ^{-1}}(X-\mu)$ حالا به محاسبه ژاکوبین $Y$  می پردازیم:\\

\[D(Y(X)) = \sqrt{\Sigma^{-1}}\]
\[J_{Y} = det(\sqrt{\Sigma^{-1}}) =\sqrt{det(\Sigma^{-1})} = \frac{1}{\sqrt{det(\Sigma)}} \]
% میدانیم $\Sigma = P^{T}DP$ پس:
% \[det(\Sigma) = det(P^{T}DP) = det(P^{T})det(D)det(P) = det(P^{-1})det(P)det(D) = det(D)\]
% پس داریم:
% $\sqrt{det(\Sigma)} = \sqrt{det(D)}$ و از طرف دیگر مشابهه قسمت قبل
% $det(\sqrt{\Sigma}) = det(\sqrt{D})$ و از آنجایی که ماتریس $D$ ماتریسی قطری است داریم:
% $\sqrt{det(D)} = det(\sqrt{D})$ پس تساوی قسمت قبل را کامل میکنیم و داریم: \\
% \[J_{Y} = det(\sqrt{\Sigma^{-1}}) = det(\sqrt{\Sigma}) = {det(\Sigma)}^{\frac{1}{2}} \]
پس در ادامه با تغییر متغیر در انتگرال داریم:\\
\[A \int_{\mathbb{R}^{n}} e^{-\frac{1}{2} Y^{T}Y} {det(\Sigma)}^{\frac{1}{2}}\, d{Y} = \frac{1}{(2\pi)^{\frac{n}{2}}}\int_{\mathbb{R}^{n}} e^{-\frac{1}{2} Y^{T}Y} \, d{Y}\]
که داریم :\\
\[ \frac{1}{(2\pi)^{\frac{n}{2}}}\int_{\mathbb{R}^{n}} e^{\sum_{i = 1}^{n}{-\frac{1}{2}Y_{i}^2} } \, d{Y} \]
و از انجایی که میدانیم $\int_{\mathbb{R}}{e^{-\frac{1}{2}Y_{i}^2}}d{Y_{i}} = (2\pi)^{\frac{1}{2}}$ در نتیجه داریم:
\[ \int_{\mathbb{R}^{n}} e^{\sum_{i = 1}^{n}{-\frac{1}{2}Y_{i}^2} } \, d{Y} = (2\pi)^{\frac{n}{2}}\] پس :\\

\[ \frac{1}{(2\pi)^{\frac{n}{2}}}\int_{\mathbb{R}^{n}} e^{\sum_{i = 1}^{n}{-\frac{1}{2}Y_{i}^2} } \, d{Y}  = \frac{(2\pi)^{\frac{n}{2}}}{(2\pi)^{\frac{n}{2}}} = 1\]

