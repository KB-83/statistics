\parte{}
اگر توزیع حاشیه ای نسبت به متغیر $X_i$  را داشته باشیم میتوانیم امید ریاضی آن را محاسبه کنیم و اما با توجه به نتیجه ای که قسمت قبل گرفتیم
توزیع حاشیه ای نسبت به دو متغیر $X_i$ و $X_j$ برابر است با:\\
\[ 
    \frac{1}{\sqrt{(2\pi)^2 det(\tilde{\Sigma})}} \text{exp}(-\frac{1}{2}{(\tilde{X} - \tilde{\mu})}^{T}\Tilde{\Sigma}^{-1}(\tilde{X} - \tilde{\mu}))
\]
که $\tilde{X} = (X_i,X_j)$ و $\tilde{\mu} = (\mu_i,\mu_j)$ و $\Sigma = 
\begin{bmatrix}
    \sigma_{ii} & \sigma_{ij} \\
    \sigma_{ij} & \sigma_{jj} \\
\end{bmatrix}
$.\\
حال اگر نسبت به $X_j$ از این تابع انتگرال بگیریم دوباره با توجه به قسمت قبل تابع توزیع حاشیه ای برای $X_i$ را خواهیم داشت
فقط به این نکته توجه کنیم که $\tilde{\Sigma} = det(\tilde{\Sigma}) = \sigma_{ii}$ پس داریم:\\
\[ 
    f_{X_i}(x_i) =  \frac{1}{\sqrt{(2\pi)\sigma_{ii}}} \text{exp}(-\frac{(X_i - \mu_i)^2}{2\sigma_{ii}})
\]
\\
 که میدانیم تابع چگالی متغیر تصادفی نرمال با میانگین $\mu_i$ است پس:
 \[ E[X_i] = \mu_i\]
\parte{}
طبق استدلال قسمت قبل و تابع چگالی بدست آمده به وضوح داریم:\\
\[\sigma_{ii} = \sigma_{i}^2 = var(X_i) \]



\parte{}
طبق قسمت $1$ داریم:\\
\[
    f_{X_i,X_j}(x_i,x_j)=\frac{1}{\sqrt{(2\pi)^2 det({\Sigma})}} \text{exp}(-\frac{1}{2}{({X} - {\mu})}^{T}{\Sigma}^{-1}({X} -{\mu}))
\]
و از طرفی کافیست $E[X_i X_j]$ 
را محاسبه کنیم چون در گام های قبل $E[X_i]$
و $E[Xـj]$
را محاسبه کردیم.
\[
    E[X_i X_j] = \int{x_i x_j f_{X_i,X_j}(x_i,x_j)} dx_i dx_j 
\]
که برای سادگی در نظر میگیریم:\\
\[
    A = \frac{1}{2\pi\sqrt{det({\Sigma})}}
\]
\[
    \Sigma = \begin{bmatrix}
        \sigma_{i}^2 & \sigma_{ij} \\
        \sigma_{ij} & \sigma_{j}^2 \\
    \end{bmatrix}  
    \quad
    \Sigma^{-1} = \begin{bmatrix}
        a & b \\
        c & d \\
    \end{bmatrix}  
\]
\\

\[
    E[X_i X_j] = A\int_{-\infty}^{\infty}{\int_{-\infty}^{\infty}
    {x_i x_jexp[a(x_i-\mu_i)^2+c(x_j-\mu)^2+(b+d)(x_i-\mu_i)(x_j-\mu_j)]}}
    dx_i dx_j 
\]

\[
    = A\int_{-\infty}^{\infty}{x_j exp[c(x_j-\mu)^2]\int_{-\infty}^{\infty}
    {x_i exp[a(x_i-\mu_i)^2+(b+d)(x_i-\mu_i)(x_j-\mu_j)]}}
    dx_i dx_j 
\]\\
ابتدا انتگرال زیر را محاسبه میکنیم:\\

\[
    \int_{-\infty}^{\infty}
    {x_i exp[a(x_i-\mu_i)^2+(b+d)(x_i-\mu_i)(x_j-\mu_j)]}
    dx_i
\]\\
در نظر میگیریم $u=(x_i-\mu_i)$ و $s = (b+d)(x_j-\mu_j)$
داریم:\\
\[
    = \int_{-\infty}^{\infty}
    {(u+\mu_i) exp[au^2+su]}
    du
    = \int_{-\infty}^{\infty}
    {(u+\mu_i) exp[a(u+s/2a)^2-s^2/4a]}
    du
\]\\
در نظر میگیریم $v=(u+s/2a)$
داریم:\\

\[
    = \int_{-\infty}^{\infty}
    {(v-\frac{s}{2a}+\mu_i) exp[av^2-s^2/4a]}
    dv
\]
\[
    = \int_{-\infty}^{\infty}
    {(v)exp[av^2-s^2/4a]}
    dv
    +
    \int_{-\infty}^{\infty}
    {(\mu_i-\frac{s}{2a}) exp[av^2-s^2/4a]}
    dv
\]\\
که از آنجایی که تابع $f(v) = (v)exp[av^2-s^2/4a]$
تابعی فرد است پس انتگرال آن روی بازه متقارن صفر و داریم:\\
\[
    = (\mu_i-\frac{s}{2a})e^{-\frac{s^2}{4a}}
    \int_{-\infty}^{\infty}
    { e^{av^2}}
    dv  
\]
اگر $a>0$ انتگرال بالا واگراست پس با شرط $a<0$ کار را ادامه میدهم داریم:\\
\[
    = (\mu_i-\frac{s}{2a})e^{-\frac{s^2}{4a}}
    \frac{\sqrt{\pi}}{\sqrt{-a}}
\]
پس برای ادامه کار داریم:\\
\[
    E[X_i X_j] = \int_{-\infty}^{\infty}{x_j exp[c(x_j-\mu)^2] (\mu_i-\frac{s}{2a})e^{-\frac{s^2}{4a}}
    \frac{\sqrt{\pi}}{\sqrt{-a}}}
    dx_j
\]\\
\[
    = \int_{-\infty}^{\infty} (\mu_i-\frac{(b+d)(x_j-\mu_j)}{2a}) {x_j exp[c(x_j-\mu)^2-\frac{((b+d)(x_j-\mu_j))^2}{4a}]
    \frac{\sqrt{\pi}}{\sqrt{-a}}}
    dx_j
\]\\
\[
    = \int_{-\infty}^{\infty} (\mu_i-\frac{s}{2a}) {x_j exp[c\frac{s^2}{(b+d)^2}-\frac{s^2}{4a}]
    \frac{\sqrt{\pi}}{\sqrt{-a}}}
    \frac{ds}{(b+d)}
\]
\\\\
% (ناقص)
