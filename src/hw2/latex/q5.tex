\problem{}
فرض می‌کنیم \( X \) متغیر تصادفی هندسی است که 
نشان دهنده اولین دفعه ای است که سکه را پرتاب میکنیم و شیر می آید و دوبار بعد از آن نیز شیر می آید.
به این ترتیب داریم:
\[ P(X = i) = (1-x)^{i-1} x \]
که در آن $x$ احتمال رخ دادن سه شیر پشت سر هم است که برابر است با $p^3$
 و میدانیم تعداد دفعات پرتاب برای رسیدن به اولین بار سه شیر پشت سر هم برابر است با :
 \[ X+2 \] 
 حالا میخواهیم متوسط تعداد دفعات پرتاب را محاسبه کنیم که یعنی $E[X+2]$ که داریم :
\[ E[X+2] = 2+E[X] = 2 + \frac{1}{p^3} \] 
علت اینکه $E[X]$ برابر با $\frac{1}{p^3}$ شد این است که امید ریاضی متغیر تصادفی هندسی با پارامتر $p$ برابر با $\frac{1}{p}$ است.
در نتیجه متوسط تعداد دفعات پرتاب برای رسیدن به این منظور برابر با $2 + \frac{1}{p^3}$ بار است.